\section{Feature Detection}
Introduction
\subsection{Color}
We first represent the color of the images in the RGB color space 
and then detect their hue, saturation, and value, i.e., their 
representation in the HSV color space.  Hue is what is typically
think of as the color of the object, i.e., the object is red,
green, purple, or etc.  Saturation is used to specify how little
or how much there is of a color relative to its brightness.  For
example, the color pink can range from a light pink (low saturation)
to a dark pink (high saturation).  Value is also known as 
brightness and represents the largest component of a color.

We represent hue using a polar plot (or color circle as it was
referred to in~\cite{preucil}).  In order to 
compute the hue angle from the RGB representation of the color,
first determine the relative ordering of the red, green, and blue
components of the color.  The hue angle can then be determined from 
use of the appropriate formula in Table~\ref{tab:hue}~\cite{preucil}.
The range for the hue is from $0$ to $359^{o}$.

\begin{table*} 
\centering   
\begin{tabular}{| c | c | l |}   \hline
{\bf{Relative Ordering}} & {\bf{Color}} & {\bf{Hue Angle}} \\ \hline
$R \geq G \geq B$ &  Red to Yellow & $h_{Preucil \ circle} = 60^{o} \cdot \frac{G-B}{R-B}$  \\ \hline
$G > R \geq B$ & Yellow to Green & $h_{Preucil \ circle} = 60^{o} \cdot \left(1+\frac{R-B}{G-B}\right)$ \\ \hline
$G \geq B > R$ & Green to Cyan & $h_{Preucil \ circle} = 60^{o} \cdot \left(2 + \frac{B-R}{G-R} \right)$\\ \hline
$B > G > R$ & Cyan to Blue & $h_{Preucil \ circle} = 60^{o} \cdot \left(3 + \frac{G-R}{B-R} \right)$\\ \hline
$B > R \geq G$ & Blue to Magenta & $h_{Preucil \ circle} = 60^{o} \cdot \left(4 + \frac{R-G}{B-G}\right)$ \\ \hline
$R \geq B > G$ & Magenta to Red & $h_{Preucil \ circle} = 60^{o} \cdot \left(5 + \frac{B-G}{R-G}\right)$ \\ \hline
\end{tabular}   
\caption{Formulas used to convert RGB color into
hue for the HSV color space~\cite{preucil}.}  
\label{tab:hue} 
\end{table*} 

The value (or brightness) of the color is determined by computing its 
largest component as follows: $$\max \{ R, G, B\}.$$  The range for the 
value is between $0$ and $1$.

The saturation of the color is a function of its value and its chroma (i.e,
its colorfulness relative to the brightness of a similarly illuminated
white).  The chroma of a color is given by $$C = 
\sqrt{\alpha^2+\beta^2},$$
where $$\alpha = \frac{1}{2} \left(2R-G-B\right)$$ and $$\beta = \frac{\sqrt{3}}{2} \left(G-B\right).$$
Then, assuming the chroma has been scaled such that $C \ \in \ [0,1]$, the saturation
for the HSV color space is given as
$$S = \begin{cases}
0 & \mbox{if } $C = 0$ \\
\frac{C}{V}, & \mbox{otherwise.}
\end{cases}
$$ 

% We're not doing texture and noise this time around.
%\subsection{Structure (Texture+Noise)}
\subsection{Geometry (Shape)}
Similar to our work in~\cite{hydrocephalus2,hydrocephalus1}, We use 
the Chan and Vese level set method~\cite{chan_vese} in order to determine 
the geometry of shapes in an image.  This level set method evolves let 
set curves using minimiation of a Mumford-Shah type energy functional.

As in~\cite{hydrocephalus1}, let $u_0$ denote an image on domain $\Omega$,
and let $C$ be a parametrized curve.  Let $\phi$ be an implicit representation
of $C$ and a Lipschitz function.  Then, the zero-level curve of $\phi(t,x,y)$
at time $t$ is used to evolve the curve $C$ based on a prescribed speed
and direction.

The evolution of the curve is obtained my minimizing the following
regularized energy functional:
\begin{eqnarray}
\begin{array}{rcl}
F_{\epsilon}(c_1,c_2,\phi) & = & \mu \int_{\Omega} \delta_{\epsilon}(\phi(x,y))|\nabla \phi(x,y)| \
dx \ dy \\
 & & + \nu \int_{\Omega} H_{\epsilon}(\phi(x,y)) \ dx \ dy \\
 & & + \lambda_1 \int_{\Omega} |u_0(x,y)-c_1|^2 \ H_{\epsilon} (\phi(x,y)) \ dx \ dy \\
 & & + \lambda_2 \int_{\Omega} |u_0(x,y)-c_2|^2 \ (1-H_{\epsilon}(\phi(x,y))) \ dx \ dy,
\end{array}
\label{eq:energy}
\end{eqnarray}
where $c_1$ and $c_2$ are constants which depend on $C$ and denote the averages of
$u_0$ on the inside and outside of $C$.  In addition, $\mu \geq 0, \nu \geq 0,
\lambda_1$, and $\lambda_2$ are constants, and $H_{\epsilon}$, $\delta_{\epsilon}$
are regularization parameters.

We minimize (\ref{eq:energy}) using a C++ implementation of Wu's 
Matlab code in~\cite{wu2011}.  See~\cite{chan_vese} for more details.

\subsection{Motion}
In order to assess the motion in the images, we perform shape matching by 
following the procedure in~\cite{jibum_kim_isvc_2010} which builds 
on~\cite{felsenzwalb}.  In particular, we first create a coarse 
constrained Delaunay triangulation of the source mesh.  Second, 
we find the optimal mapping from the source to the target image which 
minimizes our energy function.  This allows us to determine which shape 
in the source image matches which shape in the target image.

In order to generate a triangular mesh of the source image, we first 
generate an optimal polygonal approximation, $P$, of the boundary curve in 
the source image, $S$, using our optimal edge grid generator~\cite{optimal_edge_grids}.

Next, we use the constrained Delauanay triangulation method~\cite{cdt_1} 
implemented in Triangle~\cite{triangle_1} to generate a triangular mesh, 
$M$, with boundary $P$ of $S$.  It is important to note that we do not add 
any interior vertices when generating the constrained Delaunay 
triangulation.  It was noted in~\cite{felsenzwalb} that such a mesh can be 
represented using a dual graph whose vertices can be ordered and 
eliminated using a perfect elimination scheme~\cite{perfect_1}.  

Finally, we use the formulation of the optimization problem for shape 
matching in~\cite{jibum_kim_isvc_2010}.  The shape matching problem 
describes how to map each shape in $S$ to the corresponding shape in the 
target image, $T$.  In particular, the solution of the optimization 
problem prescribes where to place the boundary vertices of $M$ onto the 
corresponding shape in $T$.

Briefly, the energy function for shape matching 
in~\cite{jibum_kim_isvc_2010} consists of three terms:  an edge cost, a 
triangular deformation cost, and a triangular edge length distortion cost.

The edge cost for the $i^{th}$ triangle, $E_{edge,i}$, acts as an edge 
detector and  is given by 
$$
E_{edge,i}=\frac{1}{\lambda +\left |  \triangledown I \right |},
$$
where $\lambda$ is a constant.

The triangular deformation cost, $E_{def,i}$, measures how far the 
original triangle is transformed from a similarity 
transformation~\cite{widrow_1}.  The triangular deformation cost
is given by
$$
E_{def,i}=log^{2}\left ( \frac{\alpha }{\beta } \right ),
$$
where $\alpha$ and $\beta$ are the singular values of the matrix 
associated with an affine transformation which maps a triangle from $S$ to 
a triangle in $T$.  

Finally, the triangular edge length distortion cost, $E_{dis,i}$, is a 
term which penalizes distortions of the edge lengths of a triangle.  Let
$l_{sum,s}$ and $l_{sum_t}$ bet the sums of the edge lengths of a triangle 
in $S$ and $T$, respectively.  Now, define $\gamma$ as 
$$
\gamma =\left\{\begin{matrix}
l_{sum, s}/l_{sum, t} \  \ \mathrm{if} \   l_{sum, s} >   l_{sum, t} \\
l_{sum, t}/l_{sum. s} \  \ \mathrm{if} \   l_{sum, t} >  l_{sum, s}.
\end{matrix}\right.
$$

We also use the center of mass in the image to make the triangular edge 
length distortion cost invariant under uniform scaling of the source image 
to the target image.  Define $d_{max, s}$ and $d_{max, t}$ to be the 
maximum distances from the center of masses to the boundary vertices in 
$S$ and $T$, respectively.

Now, define $\delta$ as follows
$$
\delta =\left\{\begin{matrix}
d_{max,s} / d_{max,t} \ \ \mathrm{if} \ \ d_{max,s} > d_{max,t}\\
d_{max,t} / d_{max,t} \ \ \mathrm{if} \ \ d_{max,s} < d_{max,t}.
\end{matrix}\right.
$$

Finally, $E_{dis, i}$ is defined as follows:
$$
E_{dis,i}=\left\{\begin{matrix}
log(\gamma / \delta ) \ \ \mathrm{if} \ \ \gamma > \delta \\
log(\delta / \gamma ) \ \ \mathrm{if} \ \ \gamma < \delta.
\end{matrix}\right.
$$

{\bf{Suzanne:  This is a place holder for now.  Eventually this will be 
combined with the formulation of one large optimization problem for 
color, geometry, and motion.}}

The energy function, $E(f)$, for shape matching is given by
\begin{equation}
E(f)=\sum_{i=1}^{N}E_{i}=\sum_{i=1}^{N}(E_{edge,i}+a*E_{def,i}+b*E_{dis,i}),
\end{equation} 
where $N$ is the number of triangles in $M$ and $a$ and $b$ are scalars. 
