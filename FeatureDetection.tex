\section{Feature Detection}
Introduction
\subsection{Color}
We first represent the color of the images in the RGB color space 
and then detect their hue, saturation, and value, i.e., their 
representation in the HSV color space.  Hue is what is typically
think of as the color of the object, i.e., the object is red,
green, purple, or etc.  Saturation is used to specify how little
or how much there is of a color relative to its brightness.  For
example, the color pink can range from a light pink (low saturation)
to a dark pink (high saturation).  Value is also known as 
brightness and represents the largest component of a color.

We represent hue using a polar plot (or color circle as it was
referred to in~\cite{preucil}).  In order to 
compute the hue angle from the RGB representation of the color,
first determine the relative ordering of the red, green, and blue
components of the color.  The hue angle can then be determined from 
use of the appropriate formula in Table~\ref{tab:hue}~\cite{preucil}.
The range for the hue is from $0$ to $359^{o}$.

\begin{table*} 
\centering   
\begin{tabular}{| c | c | l |}   \hline
{\bf{Relative Ordering}} & {\bf{Color}} & {\bf{Hue Angle}} \\ \hline
$R \geq G \geq B$ &  Red to Yellow & $h_{Preucil \ circle} = 60^{o} \cdot \frac{G-B}{R-B}$  \\ \hline
$G > R \geq B$ & Yellow to Green & $h_{Preucil \ circle} = 60^{o} \cdot \left(1+\frac{R-B}{G-B}\right)$ \\ \hline
$G \geq B > R$ & Green to Cyan & $h_{Preucil \ circle} = 60^{o} \cdot \left(2 + \frac{B-R}{G-R} \right)$\\ \hline
$B > G > R$ & Cyan to Blue & $h_{Preucil \ circle} = 60^{o} \cdot \left(3 + \frac{G-R}{B-R} \right)$\\ \hline
$B > R \geq G$ & Blue to Magenta & $h_{Preucil \ circle} = 60^{o} \cdot \left(4 + \frac{R-G}{B-G}\right)$ \\ \hline
$R \geq B > G$ & Magenta to Red & $h_{Preucil \ circle} = 60^{o} \cdot \left(5 + \frac{B-G}{R-G}\right)$ \\ \hline
\end{tabular}   
\caption{Formulas used to convert RGB color into
hue for the HSV color space~\cite{preucil}.}  
\label{tab:hue} 
\end{table*} 

The value (or brightness) of the color is determined by computing its 
largest component as follows: $$\max \{ R, G, B\}.$$  The range for the 
value is between $0$ and $1$.

The saturation of the color is a function of its value and its chroma (i.e,
its colorfulness relative to the brightness of a similarly illuminated
white).  The chroma of a color is given by $$C = 
\sqrt{\alpha^2+\beta^2},$$
where $$\alpha = \frac{1}{2} \left(2R-G-B\right)$$ and $$\beta = \frac{\sqrt{3}}{2} \left(G-B\right).$$
Then, assuming the chroma has been scaled such that $C \ \in \ [0,1]$, the saturation
for the HSV color space is given as
$$S = \begin{cases}
0 & \mbox{if } $C = 0$ \\
\frac{C}{V}, & \mbox{otherwise.}
\end{cases}
$$ 

% We're not doing texture and noise this time around.
%\subsection{Structure (Texture+Noise)}
\subsection{Geometry (Shape)}
\subsection{Motion}
