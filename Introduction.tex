\section{Introduction}
Methods for image segmentation are numerous but usually only apply to a specific feature or implement a specific strategy relevant to the image features that are desired. There have been efforts to combine current segmentation tools with the goal of creating a more general tool \cite{fergus03} [OTHERS]. However, this work seeks to provide a higher level of automation to the image segmentation process by combining feature detection, multi-objective optimization, and machine learning in order to segment a large collection of similar images.

This was done by first identifying orthogonal components that can define image features: color, structure (texture+noise), geometry (shape), and motion. The authors feel that these components define a nearly comprehensive set of identifying characteristics by which images can be segmented. Each of these components was decomposed further into their respective fundamental representations. For example, color can be decomposed based on the space that is representing color in the image. RGB color space was chosen here. Other options were [choices] [REFERENCES].

A multi-objective optimization technique was used for determining an optimum combination of the above components for segmenting a single image. The next step is to use our multi-objective optimization technique for matching the segmented shapes in the initial image to the most similar shapes in the second image in the data set. Optimal shapes are determined based upon color, structure, geometry, and motion.

This process is then repeated in order to segment the remaining images in the data set. Machine learning is used as the optimization process is ongoing in order to determine the most appropriate values for the weights of the relevant terms in the objective function for a set of images.
